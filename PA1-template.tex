% Options for packages loaded elsewhere
\PassOptionsToPackage{unicode}{hyperref}
\PassOptionsToPackage{hyphens}{url}
%
\documentclass[
]{article}
\usepackage{lmodern}
\usepackage{amssymb,amsmath}
\usepackage{ifxetex,ifluatex}
\ifnum 0\ifxetex 1\fi\ifluatex 1\fi=0 % if pdftex
  \usepackage[T1]{fontenc}
  \usepackage[utf8]{inputenc}
  \usepackage{textcomp} % provide euro and other symbols
\else % if luatex or xetex
  \usepackage{unicode-math}
  \defaultfontfeatures{Scale=MatchLowercase}
  \defaultfontfeatures[\rmfamily]{Ligatures=TeX,Scale=1}
\fi
% Use upquote if available, for straight quotes in verbatim environments
\IfFileExists{upquote.sty}{\usepackage{upquote}}{}
\IfFileExists{microtype.sty}{% use microtype if available
  \usepackage[]{microtype}
  \UseMicrotypeSet[protrusion]{basicmath} % disable protrusion for tt fonts
}{}
\makeatletter
\@ifundefined{KOMAClassName}{% if non-KOMA class
  \IfFileExists{parskip.sty}{%
    \usepackage{parskip}
  }{% else
    \setlength{\parindent}{0pt}
    \setlength{\parskip}{6pt plus 2pt minus 1pt}}
}{% if KOMA class
  \KOMAoptions{parskip=half}}
\makeatother
\usepackage{xcolor}
\IfFileExists{xurl.sty}{\usepackage{xurl}}{} % add URL line breaks if available
\IfFileExists{bookmark.sty}{\usepackage{bookmark}}{\usepackage{hyperref}}
\hypersetup{
  pdftitle={PA1-Template.rmd},
  pdfauthor={kvsubbaiah},
  hidelinks,
  pdfcreator={LaTeX via pandoc}}
\urlstyle{same} % disable monospaced font for URLs
\usepackage[margin=1in]{geometry}
\usepackage{color}
\usepackage{fancyvrb}
\newcommand{\VerbBar}{|}
\newcommand{\VERB}{\Verb[commandchars=\\\{\}]}
\DefineVerbatimEnvironment{Highlighting}{Verbatim}{commandchars=\\\{\}}
% Add ',fontsize=\small' for more characters per line
\usepackage{framed}
\definecolor{shadecolor}{RGB}{248,248,248}
\newenvironment{Shaded}{\begin{snugshade}}{\end{snugshade}}
\newcommand{\AlertTok}[1]{\textcolor[rgb]{0.94,0.16,0.16}{#1}}
\newcommand{\AnnotationTok}[1]{\textcolor[rgb]{0.56,0.35,0.01}{\textbf{\textit{#1}}}}
\newcommand{\AttributeTok}[1]{\textcolor[rgb]{0.77,0.63,0.00}{#1}}
\newcommand{\BaseNTok}[1]{\textcolor[rgb]{0.00,0.00,0.81}{#1}}
\newcommand{\BuiltInTok}[1]{#1}
\newcommand{\CharTok}[1]{\textcolor[rgb]{0.31,0.60,0.02}{#1}}
\newcommand{\CommentTok}[1]{\textcolor[rgb]{0.56,0.35,0.01}{\textit{#1}}}
\newcommand{\CommentVarTok}[1]{\textcolor[rgb]{0.56,0.35,0.01}{\textbf{\textit{#1}}}}
\newcommand{\ConstantTok}[1]{\textcolor[rgb]{0.00,0.00,0.00}{#1}}
\newcommand{\ControlFlowTok}[1]{\textcolor[rgb]{0.13,0.29,0.53}{\textbf{#1}}}
\newcommand{\DataTypeTok}[1]{\textcolor[rgb]{0.13,0.29,0.53}{#1}}
\newcommand{\DecValTok}[1]{\textcolor[rgb]{0.00,0.00,0.81}{#1}}
\newcommand{\DocumentationTok}[1]{\textcolor[rgb]{0.56,0.35,0.01}{\textbf{\textit{#1}}}}
\newcommand{\ErrorTok}[1]{\textcolor[rgb]{0.64,0.00,0.00}{\textbf{#1}}}
\newcommand{\ExtensionTok}[1]{#1}
\newcommand{\FloatTok}[1]{\textcolor[rgb]{0.00,0.00,0.81}{#1}}
\newcommand{\FunctionTok}[1]{\textcolor[rgb]{0.00,0.00,0.00}{#1}}
\newcommand{\ImportTok}[1]{#1}
\newcommand{\InformationTok}[1]{\textcolor[rgb]{0.56,0.35,0.01}{\textbf{\textit{#1}}}}
\newcommand{\KeywordTok}[1]{\textcolor[rgb]{0.13,0.29,0.53}{\textbf{#1}}}
\newcommand{\NormalTok}[1]{#1}
\newcommand{\OperatorTok}[1]{\textcolor[rgb]{0.81,0.36,0.00}{\textbf{#1}}}
\newcommand{\OtherTok}[1]{\textcolor[rgb]{0.56,0.35,0.01}{#1}}
\newcommand{\PreprocessorTok}[1]{\textcolor[rgb]{0.56,0.35,0.01}{\textit{#1}}}
\newcommand{\RegionMarkerTok}[1]{#1}
\newcommand{\SpecialCharTok}[1]{\textcolor[rgb]{0.00,0.00,0.00}{#1}}
\newcommand{\SpecialStringTok}[1]{\textcolor[rgb]{0.31,0.60,0.02}{#1}}
\newcommand{\StringTok}[1]{\textcolor[rgb]{0.31,0.60,0.02}{#1}}
\newcommand{\VariableTok}[1]{\textcolor[rgb]{0.00,0.00,0.00}{#1}}
\newcommand{\VerbatimStringTok}[1]{\textcolor[rgb]{0.31,0.60,0.02}{#1}}
\newcommand{\WarningTok}[1]{\textcolor[rgb]{0.56,0.35,0.01}{\textbf{\textit{#1}}}}
\usepackage{graphicx,grffile}
\makeatletter
\def\maxwidth{\ifdim\Gin@nat@width>\linewidth\linewidth\else\Gin@nat@width\fi}
\def\maxheight{\ifdim\Gin@nat@height>\textheight\textheight\else\Gin@nat@height\fi}
\makeatother
% Scale images if necessary, so that they will not overflow the page
% margins by default, and it is still possible to overwrite the defaults
% using explicit options in \includegraphics[width, height, ...]{}
\setkeys{Gin}{width=\maxwidth,height=\maxheight,keepaspectratio}
% Set default figure placement to htbp
\makeatletter
\def\fps@figure{htbp}
\makeatother
\setlength{\emergencystretch}{3em} % prevent overfull lines
\providecommand{\tightlist}{%
  \setlength{\itemsep}{0pt}\setlength{\parskip}{0pt}}
\setcounter{secnumdepth}{-\maxdimen} % remove section numbering

\title{PA1-Template.rmd}
\author{kvsubbaiah}
\date{5/28/2020}

\begin{document}
\maketitle

setwd(``E:/DRK/coursera/Reproducableresearch/project1'')

\hypertarget{loading-and-preprocessing-the-data}{%
\subsection{Loading and preprocessing the
data}\label{loading-and-preprocessing-the-data}}

\begin{Shaded}
\begin{Highlighting}[]
\KeywordTok{unzip}\NormalTok{(}\DataTypeTok{zipfile=}\StringTok{"activity.zip"}\NormalTok{)}
\end{Highlighting}
\end{Shaded}

\begin{verbatim}
## Warning in unzip(zipfile = "activity.zip"): error 1 in extracting from zip file
\end{verbatim}

\begin{Shaded}
\begin{Highlighting}[]
\NormalTok{data <-}\StringTok{ }\KeywordTok{read.csv}\NormalTok{(}\StringTok{"activity.csv"}\NormalTok{)}
\end{Highlighting}
\end{Shaded}

\hypertarget{what-is-mean-total-number-of-steps-taken-per-day}{%
\subsection{What is mean total number of steps taken per
day?}\label{what-is-mean-total-number-of-steps-taken-per-day}}

\begin{Shaded}
\begin{Highlighting}[]
\KeywordTok{library}\NormalTok{(ggplot2)}
\NormalTok{total.steps <-}\StringTok{ }\KeywordTok{tapply}\NormalTok{(data}\OperatorTok{$}\NormalTok{steps, data}\OperatorTok{$}\NormalTok{date, }\DataTypeTok{FUN=}\NormalTok{sum, }\DataTypeTok{na.rm=}\OtherTok{TRUE}\NormalTok{)}
\KeywordTok{qplot}\NormalTok{(total.steps, }\DataTypeTok{binwidth=}\DecValTok{1000}\NormalTok{, }\DataTypeTok{xlab=}\StringTok{"total number of steps taken each day"}\NormalTok{)}
\end{Highlighting}
\end{Shaded}

\includegraphics{PA1-template_files/figure-latex/unnamed-chunk-1-1.pdf}

\begin{Shaded}
\begin{Highlighting}[]
\KeywordTok{mean}\NormalTok{(total.steps, }\DataTypeTok{na.rm=}\OtherTok{TRUE}\NormalTok{)}
\end{Highlighting}
\end{Shaded}

\begin{verbatim}
## [1] 9354.23
\end{verbatim}

\begin{Shaded}
\begin{Highlighting}[]
\KeywordTok{median}\NormalTok{(total.steps, }\DataTypeTok{na.rm=}\OtherTok{TRUE}\NormalTok{)}
\end{Highlighting}
\end{Shaded}

\begin{verbatim}
## [1] 10395
\end{verbatim}

\hypertarget{what-is-the-average-daily-activity-pattern}{%
\subsection{What is the average daily activity
pattern?}\label{what-is-the-average-daily-activity-pattern}}

\begin{Shaded}
\begin{Highlighting}[]
\KeywordTok{library}\NormalTok{(ggplot2)}
\NormalTok{averages <-}\StringTok{ }\KeywordTok{aggregate}\NormalTok{(}\DataTypeTok{x=}\KeywordTok{list}\NormalTok{(}\DataTypeTok{steps=}\NormalTok{data}\OperatorTok{$}\NormalTok{steps), }\DataTypeTok{by=}\KeywordTok{list}\NormalTok{(}\DataTypeTok{interval=}\NormalTok{data}\OperatorTok{$}\NormalTok{interval),}
                      \DataTypeTok{FUN=}\NormalTok{mean, }\DataTypeTok{na.rm=}\OtherTok{TRUE}\NormalTok{)}
\KeywordTok{ggplot}\NormalTok{(}\DataTypeTok{data=}\NormalTok{averages, }\KeywordTok{aes}\NormalTok{(}\DataTypeTok{x=}\NormalTok{interval, }\DataTypeTok{y=}\NormalTok{steps)) }\OperatorTok{+}
\StringTok{    }\KeywordTok{geom_line}\NormalTok{() }\OperatorTok{+}\StringTok{     }\KeywordTok{xlab}\NormalTok{(}\StringTok{"5-minute interval"}\NormalTok{) }\OperatorTok{+}
\StringTok{    }\KeywordTok{ylab}\NormalTok{(}\StringTok{"average number of steps taken"}\NormalTok{)}
\end{Highlighting}
\end{Shaded}

\includegraphics{PA1-template_files/figure-latex/unnamed-chunk-2-1.pdf}

On average across all the days in the dataset, the 5-minute interval
contains the maximum number of steps?

\begin{Shaded}
\begin{Highlighting}[]
\NormalTok{averages[}\KeywordTok{which.max}\NormalTok{(averages}\OperatorTok{$}\NormalTok{steps),]}
\end{Highlighting}
\end{Shaded}

\begin{verbatim}
##     interval    steps
## 104      835 206.1698
\end{verbatim}

\hypertarget{imputing-missing-values}{%
\subsection{Imputing missing values}\label{imputing-missing-values}}

There are many days/intervals where there are missing values (coded as
\texttt{NA}). The presence of missing days may introduce bias into some
calculations or summaries of the data.

\begin{Shaded}
\begin{Highlighting}[]
\NormalTok{missing <-}\StringTok{ }\KeywordTok{is.na}\NormalTok{(data}\OperatorTok{$}\NormalTok{steps)}

\CommentTok{# How many missing}

\KeywordTok{table}\NormalTok{(missing)}
\end{Highlighting}
\end{Shaded}

\begin{verbatim}
## missing
## FALSE  TRUE 
## 15264  2304
\end{verbatim}

All of the missing values are filled in with mean value for that
5-minute interval.

\begin{Shaded}
\begin{Highlighting}[]
\CommentTok{# Replace each missing value with the mean value of its 5-minute interval}

\NormalTok{fill.value <-}\StringTok{ }\ControlFlowTok{function}\NormalTok{(steps, interval) \{}
\NormalTok{    filled <-}\StringTok{ }\OtherTok{NA}
    \ControlFlowTok{if}\NormalTok{ (}\OperatorTok{!}\KeywordTok{is.na}\NormalTok{(steps))}
\NormalTok{        filled <-}\StringTok{ }\KeywordTok{c}\NormalTok{(steps)}
    \ControlFlowTok{else}
\NormalTok{        filled <-}\StringTok{ }\NormalTok{(averages[averages}\OperatorTok{$}\NormalTok{interval}\OperatorTok{==}\NormalTok{interval, }\StringTok{"steps"}\NormalTok{])}
    \KeywordTok{return}\NormalTok{(filled)}
\NormalTok{\}}

\NormalTok{filled.data <-}\StringTok{ }\NormalTok{data}
\NormalTok{filled.data}\OperatorTok{$}\NormalTok{steps <-}\StringTok{ }\KeywordTok{mapply}\NormalTok{(fill.value, filled.data}\OperatorTok{$}\NormalTok{steps, filled.data}\OperatorTok{$}\NormalTok{interval)}
\end{Highlighting}
\end{Shaded}

Now, using the filled data set, let's make a histogram of the total
number of steps taken each day and calculate the mean and median total
number of steps.

\begin{Shaded}
\begin{Highlighting}[]
\NormalTok{total.steps <-}\StringTok{ }\KeywordTok{tapply}\NormalTok{(filled.data}\OperatorTok{$}\NormalTok{steps, filled.data}\OperatorTok{$}\NormalTok{date, }\DataTypeTok{FUN=}\NormalTok{sum)}
\KeywordTok{qplot}\NormalTok{(total.steps, }\DataTypeTok{binwidth=}\DecValTok{1000}\NormalTok{, }\DataTypeTok{xlab=}\StringTok{"total number of steps taken each day"}\NormalTok{)}
\end{Highlighting}
\end{Shaded}

\includegraphics{PA1-template_files/figure-latex/unnamed-chunk-5-1.pdf}

\begin{Shaded}
\begin{Highlighting}[]
\KeywordTok{mean}\NormalTok{(total.steps)}
\end{Highlighting}
\end{Shaded}

\begin{verbatim}
## [1] 10766.19
\end{verbatim}

\begin{Shaded}
\begin{Highlighting}[]
\KeywordTok{median}\NormalTok{(total.steps)}
\end{Highlighting}
\end{Shaded}

\begin{verbatim}
## [1] 10766.19
\end{verbatim}

Mean and median values are higher after imputing missing data. The
reason is that in the original data, there are some days with
\texttt{steps} values \texttt{NA} for any \texttt{interval}. The total
number of steps taken in such days are set to 0s by default. However,
after replacing missing \texttt{steps} values with the mean
\texttt{steps} of associated \texttt{interval} value, these 0 values are
removed from the histogram of total number of steps taken each day.

\hypertarget{are-there-differences-in-activity-patterns-between-weekdays-and-weekends}{%
\subsection{Are there differences in activity patterns between weekdays
and
weekends?}\label{are-there-differences-in-activity-patterns-between-weekdays-and-weekends}}

First, let's find the day of the week for each measurement in the
dataset. In this part, we use the dataset with the filled-in values.

\begin{Shaded}
\begin{Highlighting}[]
\NormalTok{weekday.or.weekend <-}\StringTok{ }\ControlFlowTok{function}\NormalTok{(date) \{}
\NormalTok{    day <-}\StringTok{ }\KeywordTok{weekdays}\NormalTok{(date)}
    \ControlFlowTok{if}\NormalTok{ (day }\OperatorTok\StringTok{ }\KeywordTok{c}\NormalTok{(}\StringTok{"Monday"}\NormalTok{, }\StringTok{"Tuesday"}\NormalTok{, }\StringTok{"Wednesday"}\NormalTok{, }\StringTok{"Thursday"}\NormalTok{, }\StringTok{"Friday"}\NormalTok{))}
        \KeywordTok{return}\NormalTok{(}\StringTok{"weekday"}\NormalTok{)}
    \ControlFlowTok{else} \ControlFlowTok{if}\NormalTok{ (day }\OperatorTok\StringTok{ }\KeywordTok{c}\NormalTok{(}\StringTok{"Saturday"}\NormalTok{, }\StringTok{"Sunday"}\NormalTok{))}
        \KeywordTok{return}\NormalTok{(}\StringTok{"weekend"}\NormalTok{)}
    \ControlFlowTok{else}
        \KeywordTok{stop}\NormalTok{(}\StringTok{"invalid date"}\NormalTok{)}
\NormalTok{\}}
\NormalTok{filled.data}\OperatorTok{$}\NormalTok{date <-}\StringTok{ }\KeywordTok{as.Date}\NormalTok{(filled.data}\OperatorTok{$}\NormalTok{date)}
\NormalTok{filled.data}\OperatorTok{$}\NormalTok{day <-}\StringTok{ }\KeywordTok{sapply}\NormalTok{(filled.data}\OperatorTok{$}\NormalTok{date, }\DataTypeTok{FUN=}\NormalTok{weekday.or.weekend)}
\end{Highlighting}
\end{Shaded}

Now, let's make a panel plot containing plots of average number of steps
taken on weekdays and weekends.

\begin{Shaded}
\begin{Highlighting}[]
\NormalTok{averages <-}\StringTok{ }\KeywordTok{aggregate}\NormalTok{(steps }\OperatorTok{~}\StringTok{ }\NormalTok{interval }\OperatorTok{+}\StringTok{ }\NormalTok{day, }\DataTypeTok{data=}\NormalTok{filled.data, mean)}
\KeywordTok{ggplot}\NormalTok{(averages, }\KeywordTok{aes}\NormalTok{(interval, steps)) }\OperatorTok{+}\StringTok{ }\KeywordTok{geom_line}\NormalTok{() }\OperatorTok{+}\StringTok{ }\KeywordTok{facet_grid}\NormalTok{(day }\OperatorTok{~}\StringTok{ }\NormalTok{.) }\OperatorTok{+}
\StringTok{    }\KeywordTok{xlab}\NormalTok{(}\StringTok{"5-minute interval"}\NormalTok{) }\OperatorTok{+}\StringTok{ }\KeywordTok{ylab}\NormalTok{(}\StringTok{"Number of steps"}\NormalTok{)}
\end{Highlighting}
\end{Shaded}

\includegraphics{PA1-template_files/figure-latex/unnamed-chunk-7-1.pdf}

\end{document}
